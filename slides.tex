% slides.tex

\documentclass{beamer}
\usetheme{default}

\title{Data Structures for Range-Sum Queries}
\subtitle{The Evolution of the Data Cube}
\author{Paul Butler}
\institute{
    CUMC 2010\\
    Waterloo, Ontario
}
\date{July 2010}

\usepackage{amsthm}
\usepackage{tabularx,colortbl}
\usepackage{multirow}
\usepackage{cite}

\theoremstyle{definition}
\newtheorem{myexample}{Example}
\theoremstyle{definition}
\newtheorem{mynote}{Note}

\begin{document}

\begin{frame}[plain]
    \titlepage
\end{frame}

\begin{frame}{The Range-Sum Problem: Example}
    \begin{table}[h]\footnotesize
        \begin{tabular}{ | l | l | l | l | l | l |}
        \hline
        \rowcolor{red!20!green!20!blue!20}
        \textbf{Name} & \textbf{DOB} & \textbf{City} & \textbf{Height} & \textbf{Siblings} & \textbf{Pets} \\ \hline
        Joseph Matthews & Jan 9, 1987 & Waterloo & 172 & 2 &  1 \\ \hline
        Sarah Farmer & Nov 17, 1988 & Halifax & 167 & 0 & 2 \\ \hline
        \vdots & \vdots & \vdots & \vdots & \vdots & \vdots \\ \hline
        \end{tabular}
    \end{table}

    \begin{itemize}
        \item How many people in Vancouver were born before 1990?
        \item What is the average number of siblings for people above 170 cm?
        \item What is the total number of pets owned by people 18 to 24 in Calgary?
    \end{itemize}
\end{frame}

\begin{frame}{The Range-Sum Problem: Definitions}
    \begin{definition}[Measure]
        A column whose values we want to aggregate in our queries.
    \end{definition}
    \begin{myexample}
        The \textbf{Pets} and \textbf{Siblings} columns from the last example are \textbf{measures}
    \end{myexample}
    \begin{definition}[Dimension]
        A column we want to use to select columns which belong to our aggregation.
    \end{definition}
    \begin{myexample}
        The \textbf{DOB}, \textbf{City}, and \textbf{Height} columns are \textbf{dimensions}.
    \end{myexample}
\end{frame}

\begin{frame}{Dimensions vs. Measures}
    \begin{itemize}
        \item For each column, \textbf{dimension} vs. \textbf{measure} depends on which questions you want to answer.
        \item \textit{What is the average age of people with two pets}
        \begin{itemize}
            \item \textbf{DOB} would be a \textbf{measure}
            \item \textbf{Pets} would be a \textbf{dimension}
        \end{itemize}
    \end{itemize}
\end{frame}

\begin{frame}{The Na\"{i}ve Approach}
    \begin{itemize}
        \item Store the data as a table
        \item For every row:
        \begin{itemize}
            \item If the dimension columns match our query, add the value in the measure column to a running total
        \end{itemize}
        \item Return the running total
    \end{itemize}
    \textbf{Too slow} for large amounts of data!
\end{frame}

\begin{frame}{Data Cubes}

\end{frame}

\begin{frame}{Data Cubes}
    We can aggregate the data into a multi-dimensional array.\cite{Gray96}

    \begin{table}[h]\footnotesize
        \begin{tabular} { | l | l | l | l | l | l | l | l | l |}
        \hline
        & & \multicolumn{7}{|c|}{DOB} \\ \hline
        & & $\hdots$ & 1985 & 1986 & 1987 & 1988 & 1989 & $\hdots$ \\ \hline
        \multirow{6}{*}{Height}
        & \vdots & $\ddots$ & \vdots & \vdots & \vdots & \vdots & \vdots & $\ddots$ \\
        & 165 & $\hdots$ & 192 & 342 & 558 & 56 & 591 & $\hdots$ \\
        & 166 & $\hdots$ & 325 & 275 & 707 & 855 & 484 & $\hdots$ \\
        & 167 & $\hdots$ & 487 & 326 & 363 & 193 & 350 & $\hdots$ \\
        & 168 & $\hdots$ & 326 & 363 & 193 & 350 & 422 & $\hdots$ \\
        & 169 & $\hdots$ & 438 & 456 & 550 & 385 & 412 & $\hdots$ \\
        & \vdots & $\ddots$ & \vdots & \vdots & \vdots & \vdots & \vdots & $\ddots$ \\
        \hline
        \end{tabular}
    \end{table}

    Now cells not selected by the query are ignored. \\
    Less lookups, faster queries (but still too slow).
\end{frame}

\begin{frame}{Data Cubes}
    What if we calculate partial sums?
    \begin{table}[h]\footnotesize
        \begin{tabular} { | l | l | l | l | l | l | l | l | l | l |}
        \hline
        & & \multicolumn{8}{|c|}{DOB} \\ \hline
        & & $\hdots$ & 1985 & 1986 & 1987 & 1988 & 1989 & $\hdots$ & Sum \\ \hline
        \multirow{7}{*}{Height}
        & \vdots & $\ddots$ & \vdots & \vdots & \vdots & \vdots & \vdots & $\ddots$ & \vdots \\
        & 165 & $\hdots$ & 192 & 342 & 558 & 56 & 591 & $\hdots$ & 10937 \\
        & 166 & $\hdots$ & 325 & 275 & 707 & 855 & 484 & $\hdots$ & 10998 \\
        & 167 & $\hdots$ & 487 & 326 & 363 & 193 & 350 & $\hdots$ & 11064 \\
        & 168 & $\hdots$ & 326 & 363 & 193 & 350 & 422 & $\hdots$ & 10913 \\
        & 169 & $\hdots$ & 438 & 456 & 550 & 385 & 412 & $\hdots$ & 11347 \\
        & \vdots & $\ddots$ & \vdots & \vdots & \vdots & \vdots & \vdots & $\ddots$ & \vdots \\
        & Sum & $\hdots$ & 8121 & 8255 & 8206 & 8820 & 8026 & $\hdots$ & 202169 \\
        \hline
        \end{tabular}
    \end{table}

    Queries which only mention \textit{some} dimensions run faster.
\end{frame}

\begin{frame}{Data Cubes}
    \begin{myexample}
        How many dogs are owned by people born between 1986 and 1988 (inclusive)?
    \end{myexample}

    \begin{table}[h]\footnotesize
        \begin{tabular} { | l | l | l | l | l | l | l | l | l | l |}
        \hline
        & & \multicolumn{8}{|c|}{DOB} \\ \hline
        & & $\hdots$ & 1985 & 1986 & 1987 & 1988 & 1989 & $\hdots$ & Sum \\ \hline
        \multirow{7}{*}{Height}
        & \vdots & $\ddots$ & \vdots & \vdots & \vdots & \vdots & \vdots & $\ddots$ & \vdots \\
        & 165 & $\hdots$ & 192 & 342 & 558 & 56 & 591 & $\hdots$ & 10937 \\
        & 166 & $\hdots$ & 325 & 275 & 707 & 855 & 484 & $\hdots$ & 10998 \\
        & 167 & $\hdots$ & 487 & 326 & 363 & 193 & 350 & $\hdots$ & 11064 \\
        & 168 & $\hdots$ & 326 & 363 & 193 & 350 & 422 & $\hdots$ & 10913 \\
        & 169 & $\hdots$ & 438 & 456 & 550 & 385 & 412 & $\hdots$ & 11347 \\
        & \vdots & $\ddots$ & \vdots & \vdots & \vdots & \vdots & \vdots & $\ddots$ & \vdots \\
        & Sum & $\hdots$ & 8121 & \textbf{8255} & \textbf{8206} & \textbf{8820} & 8026 & $\hdots$ & 202169 \\
        \hline
        \end{tabular}
    \end{table}
    8255 + 8206 + 8820 = \textbf{25281}
\end{frame}

\begin{frame}{Prefix-Sum Table}

\end{frame}

\begin{frame}{Relative Prefix-Sum Table}

\end{frame}

\begin{frame}{$\Delta$-tree}

\end{frame}

\begin{frame}
\bibliography{slides}{}
\bibliographystyle{plain}
\end{frame}

%\begin{frame}{The Range-Sum Problem}
%\textbf{Range queries} ``apply a given aggregation operation over selected cells where the selection is specified as contiguous ranges in the domains of some of the attributes''
%\end{frame}

\begin{frame}
\textbf{Slides} \\
\url{github.com/paulgb/cumc2010/raw/master/slides.pdf} \\
\vspace{1.5em}
\textbf{Contact} \\
pbutler@uwaterloo.ca \\
\vspace{1.5em}
\textbf{Web} \\
paulbutler.org
\end{frame}

\begin{frame}{My Slide}
A displayed formula:
\[
    \int_{-\infty}^\infty e^{-x^2} \, dx = \sqrt{\pi}
\]

An itemized list:
\begin{itemize}
    \item itemized item 1
    \item itemized item 2
    \item itemized item 3
\end{itemize}

\begin{theorem}
    In a right triangle, the square of hypotenuse equals
    the sum of squares of two other sides.
\end{theorem}

\end{frame}

\end{document}

